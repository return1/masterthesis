F�r Anf�nger ist das Erlernen einer Programmiersprache schwierig. Das liegt daran, dass im heutigen Unterricht bevorzugt mit Sprachen aus der Industrie gelehrt wird. Studenten wollen in der Industrie Jobs bekommen, und legen deshalb Wert darauf, dass gefragte Technologien Bestandteil ihrer Ausbildung sind. Die Industrie wiederum will ihren Bedarf befriedigen. Dabei wird �bersehen, dass Programmiersprachen keine Technologien selbst, sondern Werkzeuge f�r Technologien sind. Im Unterricht muss eine Programmiersprache ein Werkzeug sein, mit dessen Hilfe es m�glich ist, die fundamentalen Ideen eines Unterrichtsgegenstandes zu vermitteln, ohne in einen Unterricht �ber die Programmiersprache selbst abzudriften.

Die vorliegende Arbeit stellt Python als ein solches Werkzeug vor. Sie zeigt auf, dass Python, im Gegensatz zu heute h�ufig im Unterricht zum Einsatz kommenden Sprachen (wie C, C++ oder Java), gut f�r Anf�nger geeignet ist. Aufgrund des einfachen Zugangs k�nnen mit Python viel fr�her relevante Konzepte der Informatik und Softwareentwicklung diskutiert werden. Ein weiterer wesentlicher Vorteil ist, dass auch die Arbeit der Unterrichtenden erleichtert wird. Die Sprache ist kompakt und simpel gehalten und versucht sich dem Entwickler nicht in den Weg zu stellen. Gleichzeitig ist sie eine allgemein anerkannte Sprache und findet Verwendung in der Industrie.
