Geht es um das Aneignen von Programmierkenntnissen, ist die momentane Situation nach Ansicht des Autors nicht optimal.
Sie ist verbesserungsw�rdig. Gerade in der schnelllebigen Informatik sollte im Unterricht nicht auf Trends oder Modeerscheinungen zur�ckgegriffen werden, sondern die fundamentalen Ideen der Wissenschaft Informatik mit einem guten Konzept an die Studierenden herangetragen werden. Erste Programmierkenntnisse mit z.B. C oder Java zu vermitteln bzw. vermittelt zu bekommen, ist problematisch. Diese Arbeit hat zum Ziel, einen einfacheren Weg aufzeigen und soll begr�nden, warum erste Programmiererfahrungen mit Python didaktisch sinnvoller sind.
 


