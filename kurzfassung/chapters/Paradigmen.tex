
Python ist prim�r eine objektortientierte Sprache, unterst�tzt also das objektorientierte Softwareparadigma. Jedoch ist der Entwickler bei der Verwendung von Python an kein spezielles Paradigma gebunden. Python kann somit als \textbf{Multiparadigmen-Sprache} bezeichnet werden. 

Nach \cite{misc:placer} und \cite{misc:westbrook} sind Multiparadigmensprachen f�r den Unterricht gut geeignet. Es ist wichtig den Studenten in kein Korsett hineinzuzwingen, sondern ein breites Schema an M�glichkeiten zur kreativen Entfaltung anzubieten, denn

\begin{quote}
{\itshape "'verschiedene Menschen nehmen zur L�sung von gleichen Aufgaben unterschiedliche Sichtweisen ein, und ebenso kann es sein, dass ein Mensch f�r unterschiedliche Aufgaben verschiedene L�sungsans�tze verfolgt. F�r die wenigsten Probleme gibt es die optimale L�sung, und so empfiehlt es sich, der- oder demjenigen, der das Problem l�st, die Wahl zu �berlassen, wie er oder sie am besten mit einer Aufgabe umgeht. Daher ist es f�r den praktischen
Einsatz von Programmiersprachen notwendig, diese unterschiedlichen Sichtweisen, die zu verschiedenen L�sungsstrategien f�hren, in geeigneter Form ausdr�cken zu k�nnen."'}\cite{misc:multiparadigm}
\end{quote}

Das bietet den Studenten den Freiraum, zu Beginn ihrer Programmierkarriere mittels \textbf{eines} Werkzeugs, n�mlich der Programmiersprache,  unterschiedliche Denkweisen und Programmierans�tze zu erproben. Die Studenten k�nnen sich voll und ganz auf das Verstehen der, den Paradigmen zugrundeliegenden Prinzipien konzentrieren und begreifen, dass ein Paradigma nicht von der verwendeten Programmiersprache abh�ngt, sondern von der Sichtweise auf eine Problemstellung und deren L�sung.