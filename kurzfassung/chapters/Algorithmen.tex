Das Erlernen von grundlegenden Algorithmen ist eines der wichtigsten fachdidaktischen Ziele. Es ist nicht nur ein fach�bergreifendes Konzept, sondern auch ein essentielles Werkzeug, um komplexe Problemstellungen l�sen zu k�nnen. Dabei ist es wichtig, den Studierenden einen Algorithmus auch ausprobieren zu lassen. Das einfache Erl�utern mit nat�rlicher Sprache und Pseudocode, ist zwar als Unterst�tzung und erste Beschreibung notwendig, hat aber nicht den entsprechenden Lerneffekt. Es muss eine Programmiersprache als Hilfsmittel angewandt werden. Dazu eignet sich nat�rlich Python, da man sich durch die einfache und Pseudocode-�hnliche Syntax mehr auf die Implementierung und das Verstehen des Algorithmus konzentrieren kann, als auf das Verstehen der Programmiersprache. Diese Aussage wird durch nachfolgendes Zitat aus \cite{python:chou-pyalgs} gest�tzt.

\begin{quote}
{\itshape "'Design and analysis of algorithms are a fundamental topic in computer science and engineering education. Many algorithms courses include programming assignments to help students better understand the algorithms. Unfortunately, the use of traditional programming languages forces students to deal with details of data structures and supporting routines, rather than algorithm design. Python represents an algorithm-oriented language that has been sorely needed in education. The advantages of Python include its textbook-like syntax and interactivity that encourages experimentation."'}
\end{quote}	

\cite{python:chou-pyalgs} spricht sogar von einer \textit{algorithmus-ortientierten} Sprache, wenngleich dieser Ausdruck etwas �bertrieben erscheint. Jeder Algorithmus l�sst sich in jeder beliebigen Programmierprache implementieren und zeigt die jeweiligen verschiedenen L�sungswege eines Algorithmus auf. \cite{python:chou-pyalgs} scheint damit einfach die Eignung Pythons f�r die praktische Anwendung von Algorithmen im Unterricht unterstreichen zu wollen.