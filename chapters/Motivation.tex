%Was will diese Arbeit? Warum wird sie geschrieben? Was ist das Ziel? Was ist der Mehrwert?
%Thema? Ziel?

Geht es um das Aneignen von Programmierkenntnissen, ist die momentane Situation nach Ansicht des Autors nicht optimal.
Sie ist verbesserungsw�rdig. Gerade in der schnelllebigen Informatik sollte im Unterricht nicht auf Trends oder Modeerscheinungen zur�ckgegriffen werden, sondern die fundamentalen Ideen der Wissenschaft Informatik mit einem guten Konzept an die Studierenden herangetragen werden. Erste Programmierkenntnisse mit z.B. C oder Java zu vermitteln bzw. vermittelt zu bekommen, ist problematisch. Diese Arbeit hat zum Ziel, einen einfacheren Weg aufzeigen und soll begr�nden, warum erste Programmiererfahrungen mit Python didaktisch sinnvoller sind.

Die Programmiersprache Python soll als Programmiersprache f�r den Unterrichtseinsatz vorgestellt werden und es wird gezeigt, wie gewisse Grundkonzepte mittels dieses Werkzeugs vermittelt werden k�nnen. Weiterf�hrend wird \Zope\ als Applikationsserver, basierend auf Python, f�r fortgeschrittenere Themen der Softwareentwicklung, und vor allem f�r den praktischen Teil dieser Arbeit verwendet.

Die Arbeit hat das Ziel, die Verbreitung von Python an Schulen und Universit�ten zu unterst�tzen, um damit den Sch�lern\footnote{Wenn in dieser Arbeit die m�nnliche Form verwendet wird, sind Frauen gleicherma�en gemeint, sofern nicht explizit Gegenteiliges behauptet wird.} und Studenten den Einstieg in die Programmierung und diese selbst zu erleichtern. 

Weiters ist \Zope\ f�r den Einsatz an Universit�ten vorzustellen; der einfache Zugang zu einem Open-Source Produkt soll die Vorteile f�r Lektoren und Studenten bei fortgeschrittenen Themen der Softwareentwicklung aufzeigen.

Die vorliegende Arbeit soll dabei als Entscheidungs- und Argumentationsgrundlage an genannten Ausbildungseinrichtungen dienen k�nnen.

%It has always been Van Rossum's desire to see Python used for teaching, 
 


