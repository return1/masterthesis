Die Arbeit gliedert sich in Einleitung, zwei Hauptkapitel und abschlie�ende Diskussion. Sie ist nicht klassisch in Theorie- und Praxisteil gegliedert. Eine solche Aufteilung ist bei dieser Thematik schwer zu vollziehen, die vorgestellte Theorie wird immer wieder durch praktische Beispiele unterst�tzend erl�utert. Im zweiten Kapitel ist eine Beispielimplementierung abgehandelt.

In der Einleitung werden die Argumentationsgrundlagen f�r die Arbeit aufgebaut. Die in der Arbeit verwendeten Technologien Python und \Zope\ werden vorgestellt. 

Nach Erkl�rung allgemeiner, in der Arbeit verwendeter Begrifflichkeiten und Technologien, folgt im ersten Kapitel die Abhandlung der Programmiersprache Python. Darin wird erl�utert, warum sich Python als eine Einstiegsprogrammiersprache, und vor allem f�r Unterrichtszwecke besonders gut eignet. Danach werden Grundkonzepte des Informatikunterrichts, wie Paradigmen und Algorithmen mit Python als unterst�tzendes Werkzeug analysiert. Weiters wird behandelt, welche andere Fachgegenst�nde neben Informatik von Python profitieren k�nnen, wie das Thema Open-Source auf den Unterricht angewandt werden kann, und welche Materialien und Werkzeuge als Unterst�tzung f�r den Unterricht bzw. die Unterrichtsgestaltung zur Zeit zur Verf�gung stehen.

\Zope\ ist das Thema des zweiten Kapitels. \Zope\ ist ein Applikationsframework basierend auf Python und eignet sich f�r das Lehren fortgeschrittener Softwareentwicklungsmethoden. Darin wird untersucht, ob \Zope\ als Komponentensystem betrachtet werden kann, und wie das Framework mit \CBSE\ in Verbindung gebracht werden kann. Dass das Testen und Dokumentieren von Software im Unterricht vernachl�ssigt wird, wird ebenso behandelt, wie das Lehren eines Softwareentwicklungsprozesses. Den Abschlu� des Hauptteils bildet die Beispielimplementierung aus einem realen Projekt im Arbeitsumfeld des Autors.