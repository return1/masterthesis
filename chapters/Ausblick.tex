Gregor Lingl bietet in �sterreich die Lehrerfortbildung f�r Python an. Um den Sch�lern Kosten f�r sein Buch \cite{python:forkids} zu ersparen, m�sste es approbiert werden. Weiters k�nnten mehr Angebote in der Lehrerfortbildung organisiert werden. Kurt Winterstein, ein Lehrer an einer �sterreichischen Schule meint:

\begin{quote}
{\itshape "'Langfristig m�ssten die LehrerInnen, die sich fortbilden m�chten, mehr entlastet werden. Das gilt ganz
allgemein f�r die LehrerInnenfortbildung, weil der Schulalltag meinem Empfinden nach viel
stressiger geworden ist und die Mu�e f�r eine Fortbildung fehlt."'}\cite{python:interview1}
\end{quote}

An Universit�ten und Fachhochschulen muss gepr�ft werden, ob Lehrpl�ne angepasst werden k�nnen. Als erster Schritt sollte ein Studiengang gew�hlt werden, wo das einfacher m�glich ist. Beispielsweise ein Studiengang, wo das Programmieren ein allgemeinbildendes Fach ist, von dem wenig andere Lehrveranstaltungen abh�ngen. Gleichzeitig muss �berzeugungsarbeit geleistet werden, um Python als gute Einstiegssprache und weiterf�hrende Sprache f�r fortgeschrittene Themen in der Ausbildung vorzustellen. Vorliegende Arbeit kann als Anhaltspunkt dazu verwendet werden.

Bei \Zope\ gestaltet sich ein Versuch einfacher. Hier kann eine Projektarbeit ins Leben gerufen werden. Theoretische Teile k�nnen leicht in �berblicksveranstaltungen einflie�en. Erkenntnisse daraus sollten analysiert und �ffentlich gemacht werden, da es noch keine wissenschaftliche Arbeit zum Thema "'\Zope\ im Unterricht"' gibt.

Die Entwicklung von \emph{Grok} (siehe Kapitel \ref{kapitel:zukunftzope}) sollte beobachtet werden, um  zu erfahren, ob \emph{Grok} den Zugang, zum momentan doch komplexen Einstieg in die Komponentenarchitetur von \Zope\/, erleichtert.

Langfristig k�nnte eine softwaretechnische Ausbildung mit Python und \Zope\ alle wichtigen Konzepte der Softwareentwicklung abdecken.



%{\color{red} roadmap laut letzten G�schka meeting, todos kurz-mittel und langfristig}