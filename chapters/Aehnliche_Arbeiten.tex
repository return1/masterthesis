%Verweis, Zusammenfassung und Abgrenzung �hnlicher Arbeiten.
	
\cite{python:linkweiler} stellt die Frage: Eignet sich die Skriptsprache Python f�r schnelle Entwicklungen
im Softwareentwicklungsprozess? Dabei analysiert er Python in Syntax und Semantik und zeigt Parallelen zwischen den Anforderungen an Programmiersprachen zur schnellen Softwareentwicklung, und den Anforderungen aus fachdidaktischer Sicht. Seine Arbeit kann die eing�ngliche Frage positiv beantworten. Durch die Parallelen kommt er zu dem Schluss, dass Python f�r den Einsatz in der informatischen Bildung gut geeignet ist.
	
\cite{python:miller} erl�utert in seiner Arbeit, dass informatische Bildung, vor allem das Programmieren an sich, in Zukunft den gleichen didaktischen Wert wie "'herk�mmliches"' Lesen und Schreiben haben wird. Er beschreibt, wie so ein Unterricht aussehen kann und streicht dabei Python als pr�ferierte Sprache f�r diesen Einsatz hervor. Inspiriert wurde die Arbeit dabei von \cite{python:vanrossum}.

\cite{misc:diplinformatikat} widmet sich dem Unterrichtsfach Informatik und zeigt, wie schwer es ist, dieses Fach aufgrund der Schnelllebigkeit der Materie zu lehren. Die Arbeit untersucht den Informatikunterricht an �sterreichs Schulen im Jahr 2006. Gleichzeitig wird herausgearbeitet, wie sich Informatik von anderen Unterrichtsgegenst�nden grunds�tzlich unterscheidet.

\cite{misc:Modrow} besch�ftigt sich in seiner Arbeit mit Didaktikans�tzen in der informatischen Lehre. Er diskutiert dabei ausf�hrlich die in der vorliegenden Arbeit verwendete "'fundamentale Idee"' nach \cite{schwill1}.	

\cite{zope:reportcurriculum} zeigt, wie der Applikationsserver \Zope\ im Unterricht eingesetzt werden kann. Dabei wird allerdings Version 2 des Frameworks eingesetzt. Der Bericht zeigt den erfolgreichen Einsatz von \Zope\ innerhalb eines Webapplikationsentwicklungs-Curriculums.

%NAJA:	
%ev noch diss_gebrauchstaugliche_didaktische_Software.pdf
%und Einfuehrung_in_die Didaktik_der_Informatik.pdf