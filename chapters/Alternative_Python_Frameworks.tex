%Vorstellen anderer Python-Webframeworks (Django, Turbogears). Sehr allgemein gehalten.
%Pylons, Pocoo and Clever Harold? (siehe http://www.sqlalchemy.org/)

Es existieren diverse alternative Python Frameworks, die oft in einem Zug mit \Zope\ 3 erw�hnt werden. Dabei muss beachtet werden, dass diese Frameworks eher auf spezielle Bed�rfnisse im Web zugeschnitten sind und mehr Frontend- als Backendentwicklung unterst�tzen. Es scheint, dass die meisten dieser Frameworks auf den Web2.0-Hype aufspringen und sich dabei auf die Erstellung von Wiki, Weblog, Forum, \emph{\RSS\/} und ein \emph{\AJAX\/} gesteuertes Userinterface spezialisieren. Die folgenden Frameworks basieren alle auf Python und werden mit Objektmapper-Funktionalit�t (siehe Kapitel \ref{sec:ZODB}) zur Verf�gung gestellt.

\begin{itemize}
	\item Turbogears\footnote{Turbogears: http://www.turbogears.org/}
	\item Django\footnote{Django: http://www.djangoproject.com/}
	\item Pylons\footnote{Pylons: http://pylonshq.com/}
\end{itemize}

Allerding ist \Zope\ 3 nach Ansicht des Autors das geeignete Tool, um anhand interessanter und einfacher Applikationsentwicklung, notwendige Thematik  verst�ndlich in den Unterricht zu integrieren. Diese Themen sind Testen von Software, kollaboratives Arbeiten, wissenschaftliches Arbeiten (Nachvollziehbarkeit ist auch in der Softwareentwicklung notwendig!), Komponentenverst�ndnis und Wiederverwendbarkeit, die ihren Einzug in die Lehre vor allem in praktischer Hinsicht noch nicht gefunden haben! Obig angef�hrte Frameworks scheinen jedoch f�r einfache Webapplikationsprojekte einen einfacheren Zugang zu bieten.

