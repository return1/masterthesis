\Zope\ ist und war ein innovativer Vorreiter bei Webtechnologien. In der Implementierung von Web-applikationen liegt die gro�e St�rke des Frameworks. Gleichzeitig ist durch die einfache Handhabung ein leichter Zugang in die Materie m�glich. Die Barrieren bei L�sungen aus der Java-Welt sind durchaus h�her. Somit kann eine Verwendung in der Lehre erwogen werden.\cite{zope:futurefaasen}

Ein aktuelles Projekt der \Zope\ Community ist \emph{Grok}\footnote{Grok: http://grok.zope.org/}. Grok ist ein Ansatz, den Zugang zur \Zope\ Komponentenarchitektur weiter zu vereinfachen. Momentan ist das Projekt in Entwicklung und sollte weiter beobachtet werden.

\begin{quote}
{\itshape "'I hope with Grok we will have drastically increased the approachability of Zope 3. This will hopefully make it the web framework of choice for more web developers (Python), even given the intense competition. I also hope that Zope 3 will continue to adopt technologies from outside the Zope realm, sharing code and concepts with other Python-based web frameworks. At the same time, Zope 3's components will become more separate from a monolithic application server and will start to be reused more within other web frameworks [...] Zope 2 applications will continue to evolve using Five towards using Zope 3 technology. The platforms Zope 2 and Zope 3 will continue to merge."'}\cite{zope:futurefaasen}
\end{quote}
  
\Zope\/, vor allem \Zope\ 3, ist jedoch nur einem kleinen Teil der Entscheidungstr�ger bekannt. Das sollte ein Hauptaugenmerk der Zope Community in naher Zukunft sein. Beispielsweise ist die Zope Website schwer vernachl�ssigt, was bestimmt ein Grund ist, warum so mancher Interessierte scheitert. Das Framework und dessen M�glichkeiten, vor allem auch Erfolgsgeschichten sollten besser an die �ffentlichkeit kommuniziert werden. Einen Beitrag dazu soll diese Arbeit leisten.
