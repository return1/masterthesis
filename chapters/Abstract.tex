%siehe auch www.acm.org/sigplan/oopsla/oopsla96/how93.html,
%insbesondere Kent Becks Punkt 4
%� erster Satz: erkl�rt Problem
%� zweiter Satz: erkl�rt, warum es ein Problem
%ist
%� dritter Satz: aufsehenerregender Satz mit
%Neuigkeitswert des Papiers
%� vierter Satz: erl�utert daraus resultierende
%Implikationen

%Nicht alle Studenten sind mit den Noten ihrer
%Diplomarbeiten zufrieden. Sie erhalten die
%schlechten Zensuren nicht, weil gute Ideen
%fehlen, sondern weil die Arbeiten schlecht
%aufgeschrieben sind. Mit dieser Anleitung
%werden die Qualit�t der Diplomarbeiten und
%ihre Bewertungen deutlich steigen. Auf diese
%Weise werden nicht nur die Studenten und
%ihre Betreuer zufriedener, sondern auch das
%in den Diplomarbeiten erarbeitete Wissen
%kann besser von einer Studentengeneration
%zur n�chsten weiter gegeben werden.

\section*{Kurzfassung}
\rkopf[]{Kurzfassung}
F�r Anf�nger ist das Erlernen einer Programmiersprache schwierig. Das liegt daran, dass im heutigen Unterricht bevorzugt mit Sprachen aus der Industrie gelehrt wird. Studenten wollen in der Industrie Jobs bekommen, und legen deshalb Wert darauf, dass gefragte Technologien Bestandteil ihrer Ausbildung sind. Die Industrie wiederum will ihren Bedarf befriedigen. Dabei wird �bersehen, dass Programmiersprachen keine Technologien selbst, sondern Werkzeuge f�r Technologien sind. Im Unterricht muss eine Programmiersprache ein Werkzeug sein, mit dessen Hilfe es m�glich ist, die fundamentalen Ideen eines Unterrichtsgegenstandes zu vermitteln, ohne in einen Unterricht �ber die Programmiersprache selbst abzudriften.

Die vorliegende Arbeit stellt Python als ein solches Werkzeug vor. Sie zeigt auf, dass Python, im Gegensatz zu heute h�ufig im Unterricht zum Einsatz kommenden Sprachen (wie C, C++ oder Java), gut f�r Anf�nger geeignet ist. Aufgrund des einfachen Zugangs k�nnen mit Python viel fr�her relevante Konzepte der Informatik und Softwareentwicklung diskutiert werden. Ein weiterer wesentlicher Vorteil ist, dass auch die Arbeit der Unterrichtenden erleichtert wird. Die Sprache ist kompakt und simpel gehalten und versucht sich dem Entwickler nicht in den Weg zu stellen. Gleichzeitig ist sie eine allgemein anerkannte Sprache und findet Verwendung in der Industrie.

F�r fortgeschrittene Konzepte der Softwareentwicklung kann auf komplexere Sprachen umgestiegen werden, wobei die Studenten dabei von ihren Erfahrungen mit Python stark profitieren. Der Unterricht kann aber durchaus weiter auf Python aufbauen. So zeigt diese Arbeit, wie das Komponentenframework Zope hierbei Verwendung finden k�nnte. Dabei werden Themen wie Komponentenorientierung, Reuse, Datenbanken, das Erleben eines Softwareentwicklungsprozesses und Testen und Dokumentieren von Software erl�utert. Auch Zope, das auf Python basiert, hat den Vorteil, dass im Vergleich zu anderen Applikationsframeworks ein einfacherer Zugang schnelle Lernerfolge erm�glicht.

Vorliegende Arbeit zeigt, wie mit Python der komplette Bedarf eines auszubildenden Softwareentwicklers abgedeckt wird. Durch die Schnelllebigkeit der Technik wird es immer wichtiger, die grundlegenden Konzepte einer Wissenschaft zu beherrschen, anstatt das Erlernen eines Werkzeugs, das in der Industrie aktuell ist.

\newpage
\section*{Abstract}
\rkopf[]{Abstract}

%%%OLD VERSION
%For beginners the learning of a programming language proves to be quite difficult. This is because of the fact that today's education teaches preferentially with languages from the industry. Students want to get jobs in the industry. Therefore they demand to be sufficiently trained in these technologies. The industry again wants to satisfy its need. But programming languages are not technologies by themselves, though, they are tools for technologies. In education a programming language must be a tool, with which assistance it is possible to obtain the fundamental ideas of an education topic without getting into an instruction in the programming language itself. 

%This paper presents Python as such a tool. It points out that Python, contrary to today�s frequently in education used languages (like C,  C++ or Java) is well suitable for beginners. Due to the easy access it is possible to teach relevant concepts of computer science and software development earlier in the learning process. A further substantial advantage is relief of the instructor�s. The language is compactly and simply held without hinderung the developer. Python is a generally a well respected language and finds use in the industry. 

%For advanced concepts of software development, teaching can be switched to more complex languages, whereby the students strongly profit from their experiences with Python. Education, however, can develop quite far on Python. 

%Thus this paper shows how the component-framework Zope could be of use in this field. The following topics, such as component orientation, reuse, databases, experiencing a software development process and tests- and documenting of software are discussed. Also Zope, which is based on Python, has the advantage of easy access, thus creating the possibility of fast successes in learning. This paper shows how the complete need of a software developer is covered by teaching with Python.
%%% END OF OLD VERSION

%%% VERSION LEO STOLLWITZER
%For beginners the learning of any programming language proves to be quite difficult. This is due to the fact that today's educational system focuses primarily on languages related to industry. However, programming languages for educational purposes must be tools which help students gain fundamental understanding of a topic without the need for expert knowledge in a programming language itself.

%This paper introduces Python as one example for such a tool. It points out that, unlike languages such C,  C++ or Java widely used today, Python is highly suitable for beginners. For advanced concepts of software development, teaching can be directed to more complex languages, whereby the students strongly profit from their experience with Python. If knowledge of other languages is not required, it is also possible to stay with Python, which performs well in this situation. 
%%% EOF VERSION	

%%% COMBINED VERSION
For beginners the learning of a programming language proves to be quite difficult. This is due to the fact that today's educational system focuses primarily on languages related to industry. Students want to get jobs in the industry. Therefore they demand to be sufficiently trained in these technologies. The industry again wants to satisfy its need. But programming languages are not technologies by themselves, though, they are tools for technologies. However, programming languages for educational purposes must be such tools which help students gain fundamental understanding of a topic without the need for expert knowledge in a programming language itself. 

This paper introduces Python as one example for such a tool. It points out that, unlike languages such C,  C++ or Java widely used today, Python is highly suitable for beginners. Due to the easy access it is possible to teach relevant concepts of computer science and software development earlier in the learning process. A further substantial advantage is relief of the instructor�s. The language is compactly and simply held without hinderung the developer. Python is a generally a well respected language and finds use in the industry. 

For advanced concepts of software development, teaching can be directed to more complex languages, whereby the students strongly profit from their experience with Python. If knowledge of other languages is not required, it is also possible to stay with Python, which performs well in this situation. 

Thus this paper shows how the component-framework Zope could be of use in this field. The following topics, such as component orientation, reuse, databases, experiencing a software development process and tests- and documenting of software are discussed. Also Zope, which is based on Python, has the advantage of easy access, thus creating the possibility of fast successes in learning. This paper shows how the complete need of a software developer is covered by teaching with Python.
