Ein vieldiskutiertes Thema ist, welche Sprache sich f�r den Unterricht am besten eignet. \cite{python:CRPITV52P71-80, misc:palumbo} meint, dass der Lernerfolg von der Zeit abh�ngt, die f�r tats�chliches Programmieren aufgewendet werden kann. Es sollte keine Zeit mit sprachspezifischen Eigenheiten und Syntaxproblemen "'verschwendet"' werden. Geht es nach \cite{python:CRPITV52P71-80, misc:milbrandt}, hat eine Programmiersprache f�r Unterrichtszwecke einen einfachen Zugang. Sie sollte leicht erlernbar sein, eine klare Struktur haben und vielf�ltig einsetzbar sein. Die Sprache hat eine einfache Syntax, einfaches I/O Handling, verst�ndliche String-Manipulation, aussagekr�ftige Schl�sselw�rter und verst�ndliches Feedback im Fehlerfall.

W�hrend Pascal und Logo vor einigen Jahren noch oft im Unterricht zum Einsatz kamen, sind beide heute stark aus der Mode gekommen. Gr�nde daf�r sind sicher die mangelnde Einsatzf�higkeit in der Industrie und die, so gut wie nicht m�gliche, Verwendbarkeit der Sprache bei steigender Komplexit�t der Software.

Heute z�hlen Sprachen wie C, Java und C++ zu den beliebtesten Programmiersprachen. Das ist an der Anzahl der verf�gbaren Entwickler, Lehrg�nge und Dienstleister weltweit zu erkennen \cite{misc:TIOBE}. Studien wie \cite{misc:deRaadt}, \cite{python:industry2} und \cite{misc:stephenson} belegen, dass Java, C++ und C die Sprachen mit der gr��ten Verbreitung an Universit�ten sind.

Trotz dieser Beliebtheit (oder gerade deswegen) wird viel �ber die Tauglichkeit dieser Sprachen im Unterricht diskutiert, gerade wenn es um geeignete Sprachen f�r Programmieranf�nger geht. Die oben genannten Sprachen werden als �berladen betrachtet. Die Studierenden plagen sich eher mit der Notation, als mit dem tats�chlichen Algorithmen. \cite{python:CRPITV52P71-80} erl�utert, dass die meisten Probleme von Programmierneulingen immer dasselbe Muster zeigen:
\begin{quote}
{\itshape "'[...] construct-based problems, which make it difficult to learn the correct semantics of language constructs, and plan composition problems, which make it difficult to put plans together correctly [...] students lack the skills needed to trace the execution of short pieces of code after taken their first course on programming."'}
\end{quote}

Im Zuge einer Studie an der finnischen Universit�t in Tampere ist eine Umfrage an europ�ischen Hochschulen durchgef�hrt worden. 559 Studenten und 34 Lektoren an 6 verschiedenen Universit�ten wurden unter anderem zu deren Schwierigkeiten beim Lernen und Lehren von Programmiersprachen befragt. Die daraus abgeleiteten Folgerungen besagen:
\begin{quote}
{\itshape "'the most difficult concepts to learn are the ones that require
understanding larger entities of the program instead of just details [...] abstract concepts like pointers and memory handling are difficult
to learn [...] However, the biggest problem of novice programmers does not seem to be the understanding of basic concepts but rather learning
to apply them."'}\cite{misc:difficulties}
\end{quote}

Die Studenten haben Probleme, wenn es darum geht, den Gesamtumfang eines Programmes zu erfassen und umzusetzen, dh. wie setze ich eine mir gestellte Aufgabe mit den Konzepten um, die ich gelernt habe. Dabei darf die Syntax einer Sprache nicht im Wege stehen, da sie daran hindert eine Probleml�sung zu finden und nur neue, andere Probleme schafft. Zeiger und Speichermanipulation z�hlen zu den als sehr schwierig eingestuften Konzepten.

Die in diesem Kapitel zitierte Literatur deckt sich mit den Beobachtungen des Autors aus der eigenen Ausbildung und der Abhaltung eines C Tutoriums f�r Programmieranf�nger. Die Studenten konnten genau wiedergeben, wie sie das Problem l�sen w�rden, doch konnten sie es nicht "'zu Papier"' bringen, also als Quelltext wiedergeben. Die Syntax wurde als unnat�rlich und teilweise unverst�ndlich empfunden. Manche syntaktische Eigenheiten sind f�r den Tutor auch schwer zu erkl�ren, da die Studenten die dahinterliegenden Konzepte noch nicht verstehen k�nnen. Nahezu alle Fehler resultierten aus einem Fehler in der Syntax. Der Autor konnte dabei ein hohes Ma� an Frustration und Demotivation der Studenten im Unterricht feststellen. Die Sinnhaftigkeit des Lehrinhaltes wurde weiters angezweifelt.

Die pers�nlichen Erkenntnisse des Autors und jene der obig angef�hrten Literatur lassen Handlungsbedarf erkennen. Diese Arbeit wird im weiteren Verlauf diese Problematik behandeln und zwei Werkzeuge f�r m�glichst reibungsfreien und effektiven Unterricht vorstellen.